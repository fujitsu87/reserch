\section{問題設定}
\subsection{アップリンクマルチユーザMIMO}
本研究は,複数ユーザが基地局に情報を送るアップリンクを想定して研究を行った.ユーザ数を$K$,受信アンテナ数を$N$とし,ユーザは単一の送信アンテナを持つことを想定する.また,簡単のため,フェージング係数が観測時間$T$の間一定であるブロックフェージング通信路を仮定する.受信信号$\boldsymbol{Y}\in\mathbb{C}^{N\times T}$は式(\ref{eq:ReceiveSignal})にて与えられる.
\begin{equation} 
	\label{eq:ReceiveSignal}
	\boldsymbol{Y} = \frac{1}{\sqrt{N}}\boldsymbol{HX+W}.
\end{equation}
$\boldsymbol{X}\in\mathbb{C}^{K\times T}$は全ユーザの送信信号であり,$\boldsymbol{H}\in\mathbb{C}^{N\times K}$はすべてのユーザとアンテナ間のフェージング係数である.ここで,$\boldsymbol{H}$に関して,すべての行列成分は,互いに独立で同一の分布(independent and identically distribution : i.i.d)のレイリーフェージングに従うと仮定する.具体的には,それぞれが独立した円対称複素ガウス雑音(circularly symmetric complex Gaussian : CSCG)であり,分散は1とした.また,$\boldsymbol{W}\in\mathbb{C}^{N\times T}$は受信時に生じる雑音のことであり,それぞれが独立した円対称複素ガウス雑音で,分散は$N_0$とした.

ここで,基地局間干渉のため,ユーザーを二つのグループに分ける.一方のグループは自分の基地局のエリアに存在するユーザで,$K/2$人で構成され,残る$K/2$人のユーザは別の基地局のエリアのユーザであり,自分の基地局の信号に干渉してくる.これを踏まえ,$\boldsymbol{X}$を式(\ref{eq:SendSignal})のように定義する.
\begin{equation} 
	\label{eq:SendSignal}
	\boldsymbol{X} =  \left(
		\begin{array}{cccc}
			\boldsymbol{X}_{11} &\boldsymbol{X}_{12} &\boldsymbol{P}\\
			\boldsymbol{P} &\boldsymbol{X}_{21} &\boldsymbol{X}_{22}.
		\end{array}
	\right)
\end{equation}
$\boldsymbol{P}$は$K/2\times T_{p}$のパイロット行列であり,基地局側にとってこの信号は既知である.$\boldsymbol{X}$の行方向は観測時間$T$であるので,$\boldsymbol{P}$の送信時間の違いによって,行列(\ref{eq:SendSignal})は上半分と下半分で基地局を分けている.また,$\boldsymbol{X}$信号はそれぞれ,式(\ref{eq:QPSK})のような電力1のQPSK信号である.電力1というのは,ユーザの長期的な平均電力とする.
\begin{equation} 
	\label{eq:QPSK}
	x_{kt} = \{u+jv:u,v=\pm\sqrt{1/2}\}.
\end{equation}

\subsection{通信路推定とデータ推定}
受信側で推定するデータを$\hat{\boldsymbol{X}}$として,ここでは,推定するデータは事後平均推定
\begin{equation} 
	\label{eq:PosterorMean}
	\hat{\boldsymbol{X}}=\mathbb{E}[\boldsymbol{X}|\boldsymbol{Y},\boldsymbol{P}]
\end{equation}	
を目標とする.しかし,大規模MIMOシステムでは,式(\ref{eq:PosterorMean})を現実的な時間で解くことは,不可能である.そこで,AMPアルゴリズムを用いる.詳しい式の導出は,\ref{sec:AMP}にて説明する.AMPアルゴリズムを使う条件として,$N,K,T,Tp$が無限大に近く,$\alpha=K/N$,$\beta=T/N$,$\tau=T_{p}/T$が一定で保たれる必要がある.AMPアルゴリズムでは,表\ref{tb:Message}で示される8つのメッセージをそれぞれ交換することで推定を行っていく.初期値として,パイロット信号が入っている$(k,t) \in \{1,...,K/2\}\times \{T-T_{p}+1,...,T\}$もしくは$(k,t) \in \{K/2+1,...,K\}\times \{1,...,T_{P}\}$のとき,$\hat{x}_{kt}=x_{kt}$,$\xi_{kt}=0$となり,パイロット信号が入っていない,それ以外の成分は$\hat{x}_{kt}=0$,$\xi_{kt}=1$とした.また,$(n,t)$の要素は,$\hat{h}_{nt}=0$,$\eta_{nt}=1$とした.
\begin{table}[htb]
	\begin{center}
		\caption{AMPアルゴリズムで使用されるメッセージ \label{tb:Message}}
		\begin{tabular}{|c|c|} \hline
			$\hat{x}_{kt}$ & $x_{kt}$の事後平均 \\ \hline
			$\xi_{kt}$ & $x_{kt}$の事後分散 \\ \hline
			$\overline{x}_{kt}$ & $x_{kt}$の外部平均 \\ \hline
			$\overline{\xi}_{kt}$ & $x_{kt}$の外部分散 \\ \hline\hline
			$\hat{h}_{nk}$ & $h_{nk}$の事後平均 \\ \hline
			$\eta_{nk}$ & $h_{nk}$の事後分散 \\ \hline\hline
			$\overline{I}_{nt}$ & $y_{nt}$の干渉の平均 \\ \hline
			$\zeta_{nt}$ & $y_{nt}$の干渉の分散 \\ \hline
		\end{tabular}
	\end{center}
\end{table}

ここで,各メッセージを計算するための定義式を記す.まず,干渉を差し引いた出力$z\in\mathbb{C}$は
\begin{equation} 
	\label{eq:z}
	z_{nt}=\frac{y_{nt}-\overline{I}_{nt}}{N_{0}+\zeta_{nt}}
\end{equation}
と定義する.さらに,$\Re[x_{kt}]$の軟判定関数として,以下のような関数を定義する.
\begin{equation} 
	\label{eq:fk}
	f_{k}(u;v)=\frac{e^{2u/v}-e^{-2u/v}}{e^{2u/v}+e^{-2u/v}}.
\end{equation}
さらに,複素関数$A_{kt}(z)$として,以下のような関数を定義する.
\begin{equation} 
	\label{eq:Akt}
	A_{kt}(z)=\Re[z]\frac{\partial f_{k}}{\partial u}(\Re[\overline{x}_{kt}];\overline{\xi}_{kt})+j\Im[z]\frac{\partial f_{k}}{\partial u}(\Im[\overline{x}_{kt}];\overline{\xi}_{kt}).
\end{equation}

次に,データ推定に関わるメッセージの式を以下に示す.
\begin{equation} 
	\label{eq:x_h}
	\hat{x}_{kt}=f_{k}(\Re[\overline{x}_{kt}],\overline{\xi}_{kt})+jf_{k}(\Im[\overline{x}_{kt}],\overline{\xi}_{kt}),
\end{equation}
\begin{equation} 
	\label{eq:xi}
	\xi = 1 - |\hat{x}_{kt}|^2,
\end{equation}
\begin{equation} 
	\label{eq:x_b}
	\overline{x}_{kt} = 
		\frac{\overline{\xi}_{kt}}{\sqrt{N}}
			\sum_{n=1}^{N}\hat{h}^{*}_{nk}z_{nt}
			+\left(
				1-\frac{\overline{\xi}_{kt}}{\sqrt{N}}\sum_{n=1}^{N}\eta_{nk}|z_{nt}|^2
			\right)
			\hat{x}_{kt},
\end{equation}
\begin{equation} 
	\label{eq:xi_b}
	\overline{\xi}_{kt}=
	\left(
		\frac{1}{N}
		\sum_{n=1}^{N}
			\frac
			{|\hat{h}_{nk}|^{2}}
			{N_{0}+\zeta_{nt}}
	\right)^{-1}.
\end{equation}
次に,通信路推定に関するメッセージの式を示す.
\begin{equation}
	\label{eq:h_h}
	\hat{h}_{nk}=
	\frac
		{\eta_{nk}}
		{\sqrt{N}}
	\sum_{t=1}^{T}
		\hat{x}^{*}_{kt}z_{nt}
	+
	\left(
		1-\eta_{nk}
	\right)
	\hat{h}_{nk}
	-
	\frac
		{\eta_{nk}}
		{N}
	\sum_{t=1}^{T}
		\overline{\xi}_{kt}
		A^{*}_{kt}
		\left(
			\hat{h}_{nk}^{*}
			z_{nt}
		\right)
		z_{nt},	
\end{equation}
\begin{equation}
	\label{eq:eta}
	\left(
		1+
		\frac{1}{N}
		\sum^{T}_{t=1}
			\frac
			{|\hat{x}_{kt}|^{2}}
			{N_{0}+\zeta_{nt}}
	\right)^{-1}.
\end{equation}
最後に,干渉に関するメッセージの式を示す.
\begin{equation}
	\label{eq:I_b}
	\overline{I}_{nt}=
		\frac{1}{\sqrt{N}}
		\sum_{k=1}^{K}
			\hat{h}_{nk}\hat{x}_{kt}
		-
		\frac{1}{N}
		\sum^{K}_{k=1}
			\overline{\xi}_{kt}
			A_{kt}
			\left(
				\hat{h}_{nk}^{*}
				z_{nt}
			\right)
			\hat{h}_{nk}
		-
		\frac{1}{N}
		\sum^{K}_{k=1}
			\eta_{nk}
			|\hat{x}_{kt}|^{2}
			z_{nt},
\end{equation}
\begin{equation}
	\label{eq:zeta}
	\zeta_{nt}=
		\sum_{k=1}^{K}
			\left(
				\eta_{nk}\xi_{kt}
				+
				\eta_{nk}|\hat{x}_{kt}|^{2}
				+
				|\hat{h}_{nk}|^{2}\xi_{kt}
			\right)
\end{equation}
AMPアルゴリズムでは,式(\ref{eq:x_h})-(\ref{eq:zeta})を解くことで,通信路とデータを同時推定する.