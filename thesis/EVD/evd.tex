\documentclass[11pt]{jsarticle}
\usepackage{amsthm,amsmath,amssymb,listings,ascmac,jlisting}
\usepackage[dvipdfmx]{graphicx}


\lstset{%
  language={C},
  basicstyle={\small},%
  identifierstyle={\small},%
  commentstyle={\small\itshape},%
  keywordstyle={\small\bfseries},%
  ndkeywordstyle={\small},%
  stringstyle={\small\ttfamily},
  frame={tb},
  breaklines=true,
  columns=[l]{fullflexible},%
  % numbers=left,%
  xrightmargin=0zw,%
  xleftmargin=3zw,%
  numberstyle={\scriptsize},%
  stepnumber=1,
  numbersep=1zw,%
  lineskip=-0.5ex%
}

\makeatletter
\long\def\@makecaption#1#2{{\small
  \advance\leftskip .0628\linewidth
  \advance\rightskip .0628\linewidth
  \vskip\abovecaptionskip
  \sbox\@tempboxa{#1\hskip1zw\relax #2}%
  \ifdim \wd\@tempboxa <\hsize \centering \fi
%  #1\hskip1zw\relax #2\par
  #1{\hskip1zw\relax}#2\par
  \vskip\belowcaptionskip}}
\makeatother


\title{EVD-BASED CHANNEL ESTIMATION IN MULTICELL MULTIUSER MIMO SYSTEMS
WITH VERY LARGE ANTENNA ARRAYS}
\author{Hien Quoc Ngo Erik G. Larsson}
\date{\today}

\begin{document}
\maketitle
\section{システムモデル}
$L$個のセルを有するマルチセルMU-MIMOを考える。 各セルは、$K$個の単一アンテナユーザと、$M$個のアンテナを備えた1つのBSを含む。すべてのセルが同一周波数を使用。すべてのユーザは所望のBSに信号を伝送するアップリンクを考える。$l$番目のBSにおけるM $\times$1の受信ベクトルは以下のとおりである。
\begin{equation}
  {\bf y}_{l}=\sqrt{p_{u}}\sum_{i=1}^{L}{\bf G}_{li}{\bf x}_{i}(n)+{\bf n}_{l}(n)
\end{equation}
$\sqrt{p_{u}}{\bf x}_{i}(n)$は$i$番目のセル(各ユーザによって使用される平均電力は$p_{u}$である)内のK個のユーザによって集合的に送信されたシンボルのK×1ベクトルである。${\bf n}_{l}(n)$はM×1の加算白色ノイズであり、その要素は平均がゼロで単位分散がガウスである。${\bf G}_{li}$はl番目のBSとK個のユーザとの間のM×Kチャネル行列である。
各要素$g_{limk}\triangleq[Gli]mk$は、l番目のBSのm番目のアンテナとi番目のセルのk番目のユーザーとの間のチャネル係数であり以下の式で表される。
\begin{equation}
  g_{limk}=h_{limk}\sqrt{\beta_{lik}},m=1,2,...M
\end{equation}
$h_{limk}$は、i番目のセルのk番目のユーザからl番目のBSのm番目のアンテナへの高速フェージング係数である。$h_{limk}$は平均0分散1の確率変数。$\beta_{lik}
$は幾何学的な減衰とシャドウフェージングを表し、アンテナインデックスとは無関係であると仮定され、定数であり、先験的に知られている。
そして、チャネル行列${\bf G}_{li}$は、次式で表される。
\begin{equation}
  {\bf G}_{li}={\bf H}_{li}{\bf D}_{li}^{1/2}
\end{equation}
${\bf H}_{li}$はi番目のセルとl番目のBSのKユーザー間の高速フェージング係数M×K行列であり、$[H_{li}]_{mk} = h_{limk}$であり、$D_{li}$は対角要素が$[D_{li}]_{kk} = \beta_{lik}$であるK×K対角行列である。
\section{数学的予備知識}
受信ベクトル${\bf y}_{l}$の共分散行列の性質を考察する。 (1)と(3)から、以下の共分散行列が与えられる。
\begin{equation}
  {\bf R_{y}}\triangleq\mathbb{E}\{{\bf y_{l}y_{l}^{H}}\}=p_{u}\sum_{i=1}^{L}{\bf H}_{li}{\bf D}_{li}{\bf H}_{li}^{H}+{\bf I}_{M}
\end{equation}
多数の法則から、BSアンテナの数が多い場合、高速チャネル係数がi.i.dである場合には、 ユーザとBSとの間のチャネルベクトルは、対に直交するようになる、すなわち、
\begin{equation}
  \frac{1}{M}{\bf H}_{li}^{H}{\bf H}_{lj}\to\delta_{ij}{\bf I}_{K},as M\to\infty
\end{equation}
\bibliographystyle{ieeetr}
\bibliography{reference}
\end{document}
