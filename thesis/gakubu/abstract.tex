\chapter*{概要}
2020年東京オリンピック・パラリンピック大会に向けて,日本国内の情報通信基盤(ICT)を飛躍的に向上させる戦略が,総務省を中心として活発になっている.その戦略の一つとして,第5世代移動通信システム(5G)の実現がある.スマートフォンやIoTの拡大により,現行の無線通信規格である4G/LTEよりもさらに超高速・大容量のモバイル通信ネットワークとして,5Gの実現が求められる.

5Gの中心的役割を担う技術が,大規模MIMO(multiple-input multiple-output)である.大規模MIMO は,同一の基地局を利用する数十人のユーザを100本以上の受信アンテナを持つ基地局でサポートすることで,多入力多出力のシステムを実現し,データレートの増加やダイバーシチによる性能改善を図ることができる.しかし,大規模MIMOでは通信路推定の際,基地局間干渉によりパイロット汚染が発生する.パイロット汚染とは,直交パイロット系列の総数に関する制約のために、パイロット信号系列が短い場合に隣接する基地局間で同じ系列を利用せざるを得ず,ユーザの通信路を推定することができなくなってしまう現象である.

本研究では,ユーザから基地局へ情報伝送を行う大規模MIMOアップリンクを想定する.パイロット汚染の軽減を目的として,送信側でパイロット信号の送信タイミングを基地局ごとにずらす方法を検討する.

通信路と送信データを基地局側で同時推定するためのアルゴリズムとして,近似的メッセージ伝播法(AMP)を採用する. AMPは,人口知能分野で提案された確率伝播法を基礎として発展した反復推定法である.大規模MIMOの受信信号を通信路行列と送信データ行列との積に白色雑音を足した信号としてモデル化し,AMPアルゴリズムを使って受信信号より通信路と送信データを同時推定する.反復の手順として.パイロット信号に基づいて初期の通信路を推定し,その推定結果をもとに送信データを推定する.さらにデータの推定結果を通信路推定器にフィードバックすることで,初期の通信路推定を改善するという反復を繰り返す.

基地局数2,基地局のアンテナ数128本,基地局当たりのユーザ数8人,送信フレーム長1000,パイロット信号の長さ300,通信路/データ推定器内における反復回数30回,両者の間の反復回数3回の場合を想定して,数値シミュレーションを行った結果,信号対雑音比(SNR)が10 dBにおいて約$10^{-3}$のビット誤り率を達成できることを確認した.
