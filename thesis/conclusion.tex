\chapter{結論}
まず,通信路推定に関して説明する.パイロット汚染を抑制するために,式(\ref{eq:SendSignal})のようにパイロット信号を離して送信したが,パイロット汚染が発生しない場合,式(\ref{eq:SendSignal_all})のように基地局ごとにパイロット信号を時間的に並べて送信した方が通信路推定の精度は高い.
\begin{equation} 
	\label{eq:SendSignal_all}
	\boldsymbol{X} =  \left(
		\begin{array}{cc}
			\boldsymbol{P}_{1} &\boldsymbol{X}_{12}\\
			\boldsymbol{P}_{2} &\boldsymbol{X}_{21}.
		\end{array}
	\right)
\end{equation}
なぜならば,ユーザ同士の干渉は式(\ref{eq:I_b})が示すように,全ユーザの$\hat{x}_{kt}$がないと計算できないためである.しかし,データ推定した値を利用し,再び通信路推定を行うことで,時間シフトさせた送信データでも通信路推定を行うことができた.

次にデータ推定の結果について説明する.図(\ref{fig:bit_err})のように振動が発生している.原因として,近似的メッセージ伝搬法では,$N,K,T\to \infty$を仮定して計算式を導出しているため,有限の条件では必ず収束するとは限らないためだと考えられる\cite{kabashima}.

また,\ref{sec:contdition}と同様の条件で$\tau=0.3$の比率はそのままで,$T,Tp$を小さくした場合,推定値は発散してしまった.この原因も有限サイズのシステムによって推定精度が悪化したためだと考察される.

今後の課題として,$T,Tp$を小さくしても,推定値が発散せずに推定できるような計算式やアルゴリズムを改良することや,基地局数を増やしてシミュレーションを行うことが必要である.